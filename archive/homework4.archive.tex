\documentclass[paper=a4, fontsize=11pt]{scrartcl} % A4 paper and 11pt font size
\usepackage[top=1in, bottom=1.5in, left=1in, right=1in]{geometry}
\usepackage{fancyhdr} % Required for custom headers
\usepackage{lastpage} % Required to determine the last page for the footer
\usepackage{extramarks} % Required for headers and footers
\usepackage[usenames,dvipsnames]{color} % Required for custom colors
\usepackage{graphicx} % Required to insert images
\usepackage{listings} % Required for insertion of code
\usepackage{courier} % Required for the courier font
\usepackage{amsmath}
\usepackage[super]{nth}
\usepackage{booktabs}
\usepackage[usenames,dvipsnames]{xcolor}
\usepackage{tcolorbox}
\usepackage{tabularx}
\usepackage{array}
\usepackage{colortbl}

%\usepackage[T1]{fontenc} % Use 8-bit encoding that has 256 glyphs
%\usepackage{fourier} % Use the Adobe Utopia font for the document - comment this line to return to the LaTeX default
\usepackage[english]{babel} % English language/hyphenation
\usepackage{amsmath,amsfonts,amsthm} % Math packages
\usepackage{graphicx}

\usepackage{hyperref}
\hypersetup{
  colorlinks   = true, %Colours links instead of ugly boxes
  urlcolor     = blue, %Colour for external hyperlinks
  linkcolor    = blue, %Colour of internal links
  citecolor   = red %Colour of citations
}

\usepackage{fancyhdr} % Custom headers and footers
\pagestyle{fancyplain} % Makes all pages in the document conform to the custom headers and footers
\fancyhead{} % No page header - if you want one, create it in the same way as the footers below
\fancyfoot[L]{} % Empty left footer
\fancyfoot[C]{} % Empty center footer
\fancyfoot[R]{\thepage} % Page numbering for right footer
\renewcommand{\headrulewidth}{0pt} % Remove header underlines
\renewcommand{\footrulewidth}{0pt} % Remove footer underlines
\setlength{\headheight}{13.6pt} % Customize the height of the header
\newcommand{\ts}{\textsuperscript}

\numberwithin{equation}{section} % Number equations within sections (i.e. 1.1, 1.2, 2.1, 2.2 instead of 1, 2, 3, 4)
\numberwithin{figure}{section} % Number figures within sections (i.e. 1.1, 1.2, 2.1, 2.2 instead of 1, 2, 3, 4)
\numberwithin{table}{section} % Number tables within sections (i.e. 1.1, 1.2, 2.1, 2.2 instead of 1, 2, 3, 4)

\setlength\parindent{0pt} % Removes all indentation from paragraphs - comment this line for an assignment with lots of text

% Default fixed font does not support bold face
\DeclareFixedFont{\ttb}{T1}{txtt}{bx}{n}{8} % for bold
\DeclareFixedFont{\ttm}{T1}{txtt}{m}{n}{8}  % for normal

%----------------------------------------------------------------------------------------
%	CODE BLOCKS
%----------------------------------------------------------------------------------------

\usepackage{adjustbox}
\usepackage{listings}
\usepackage{color}

\definecolor{dkgreen}{rgb}{0,0.6,0}
\definecolor{gray}{rgb}{0.5,0.5,0.5}
\definecolor{mauve}{rgb}{0.58,0,0.82}

\lstdefinelanguage{Dockerfile}
{
  morekeywords={FROM, RUN, CMD, LABEL, MAINTAINER, EXPOSE, ENV, ADD, COPY,
    ENTRYPOINT, VOLUME, USER, WORKDIR, ARG, ONBUILD, STOPSIGNAL, HEALTHCHECK,
    SHELL},
  morecomment=[l]{\#},
  morestring=[b]"
}

\lstset{
    columns=flexible,
    aboveskip=5mm,
    belowskip=5mm,
    keepspaces=true,
    showstringspaces=false,
    basicstyle=\ttfamily,
    commentstyle=\color{gray},
    keywordstyle=\color{purple},
    stringstyle=\color{green}
}



%----------------------------------------------------------------------------------------
%	TITLE SECTION
%----------------------------------------------------------------------------------------

\usepackage{eso-pic}
% \usepackage[demo]{graphicx}
\newcommand\AtPageUpperRight[1]{\AtPageUpperLeft{%
   \makebox[\paperwidth][r]{#1}}}

\newcommand{\horrule}[1]{\rule{\linewidth}{#1}} % Create horizontal rule command with 1 argument of height

\title{	
\normalfont \normalsize
\textsc{Northeastern University,  Khoury College of Computer Science} \\ [25pt] % Your university, school and/or department name(s)
\horrule{0.5pt} \\[0.4cm] % Thin top horizontal rule
\huge CS 6220  Data Mining \textemdash~Assignment X\\ % The assignment title
\Large \textbf{DUE DATE: February 8, 2022} % The assignment title 
\horrule{2pt} \\[0.5cm] % Thick bottom horizontal rule
}

\author{\textbf{YOUR NAME} \\ \textbf{YOUR LDAP}} % Your name
\date{} % Today's date or a custom date

\begin{document}

\AddToShipoutPictureBG*{%
  \AtPageUpperRight{\raisebox{-\height}{\includegraphics[width=3cm]{images/logo}}}}
\maketitle % Print the title

%%%%%%%%%%%%%%%%%%%%
\section{Association Rules}
%%%%%%%%%%%%%%%%%%%%

Association Rules are frequently used for Market Basket Analysis (MBA) by retailers to understand the purchase behavior of their customers. This information can be then used for many different purposes such as cross-selling and up-selling of products, sales promotions, loyalty programs, store design, discount plans and many others. 

Evaluation of item sets: Once you have found the frequent itemsets of a dataset, you need to choose a subset of them as your recommendations. Commonly used metrics for measuring significance and interest for selecting rules for recommendations are: 

\begin{enumerate}
\item \textbf{Confidence} (denoted as $conf(A \rightarrow B)$ is defined as the probability of occurrence of $B$ in a basket if the basket already contains $A$:
\begin{equation}
    conf(A \rightarrow B) = P(B | A )
\end{equation}
where $P(B|A)$ is the conditional probability of finding item set $B$ given that item set $A$ is present.
\item \textbf{Lift} (denoted as $lift(A \rightarrow B)$ measures how much more ``A and B occur together'' than ``what would be expected if A and B were statistically independent'':
\begin{equation}
    lift(A \rightarrow B) = \frac{P(A, B)}{P(A) \cdot P(B)}
\end{equation}
\item \textbf{Conviction} (denoted as $conv(A \rightarrow B)$) compares the ``probability that A appears without B if they were independent'' with the ``actual frequency of the appearance of A without B''
\begin{equation}
    conv(A \rightarrow B) = \frac{1 - sup(B)}{1 - conf(A \rightarrow B)}
\end{equation}
\end{enumerate}

\subsection{Limitations of Confidence}
A drawback of using confidence is that it ignores $P(B)$. Why is this a drawback? Explain why lift and conviction do not suffer from this drawback. 

\subsection{Symmetry in Rules}
A measure is symmetrical if $measure(A \rightarrow B) = measure(B \rightarrow A)$. Which of the measures presented here are symmetrical? For each measure, please provide either a proof that the measure is symmetrical, or a counterexample that shows the measure is not symmetrical. 

\section{Clustering and Clustering Analysis}

Write a function that calculates $k$-means with unique cost functions.
    

\section{Reference}

\subsection{Market Basket Analysis}
\begin{enumerate}
    \item \url{https://www.thedataschool.co.uk/liu-zhang/understanding-lift-for-market-basket-analysis}
    \item \url{https://books.psychstat.org/rdata/market-basket-analysis.html}
\end{enumerate}

\end{document}